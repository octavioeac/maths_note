\documentclass[12pt]{article}
\usepackage{amsmath, amssymb, amsthm, geometry}
\geometry{margin=2.5cm}

\title{Apuntes: Cauchy--Schwarz, Formas Cuadráticas y Proyección Ortogonal}
\author{}
\date{}

\begin{document}
\maketitle

\section*{1. La forma cuadrática asociada}
Sea $(H,\langle \cdot,\cdot\rangle)$ un espacio de Hilbert. 
Para $f,g \in H$ definimos
\[
Q(\lambda) \;=\; \|f - \lambda g\|^2, \qquad \lambda \in \mathbb{C}.
\]

\noindent
Expandiendo con el producto interno:
\[
Q(\lambda) \;=\; \|f\|^2 - 2\Re\!\big(\lambda \langle f,g\rangle\big) + |\lambda|^2 \|g\|^2.
\]

Es decir, $Q(\lambda)$ es una \emph{parábola} en la variable $\lambda$.

\section*{2. Optimización}
Como $Q(\lambda) \geq 0$ para todo $\lambda$, el valor mínimo se alcanza en
\[
\lambda^* \;=\; \frac{\langle f,g\rangle}{\|g\|^2}.
\]

\noindent
Sustituyendo:
\[
Q(\lambda^*) \;=\; \|f\|^2 - \frac{|\langle f,g\rangle|^2}{\|g\|^2}.
\]

\section*{3. Desigualdad de Cauchy--Schwarz}
De la positividad se sigue:
\[
Q(\lambda^*) \;\geq\; 0 
\quad \Longrightarrow \quad
|\langle f,g\rangle|^2 \;\leq\; \|f\|^2 \, \|g\|^2.
\]

\noindent
Esto es precisamente la \textbf{desigualdad de Cauchy--Schwarz}.

\section*{4. Interpretación geométrica}
\begin{itemize}
  \item $Q(\lambda)$ representa la \textbf{distancia al cuadrado} entre $f$ y el subespacio generado por $g$:
  \[
  Q(\lambda) = \|f - \lambda g\|^2.
  \]
  \item El mínimo $Q(\lambda^*)$ corresponde a la \textbf{distancia ortogonal} de $f$ al subespacio $\text{span}\{g\}$.
  \item El vector proyectado es
  \[
  P_g(f) = \lambda^* g = \frac{\langle f,g\rangle}{\|g\|^2}\, g.
  \]
  \item El residuo $f - P_g(f)$ es ortogonal a $g$:
  \[
  \langle f - P_g(f), \, g \rangle = 0.
  \]
\end{itemize}

\section*{5. Continuidad}
La función $Q(\lambda)$ es cuadrática y por tanto continua en $\lambda$. 
Esto garantiza que el paso al mínimo no requiere argumentos de límite $\varepsilon$--$\delta$, 
sino que la geometría misma del espacio de Hilbert asegura la continuidad y la convergencia.


\section*{6. Aplicaciones en Machine Learning}

La desigualdad de Cauchy--Schwarz y la interpretación de la forma cuadrática 
aparecen en múltiples áreas de Machine Learning:

\subsection*{6.1 Similitud de coseno}
Para embeddings $x,y \in \mathbb{R}^n$, se define
\[
\cos(\theta) = \frac{\langle x,y\rangle}{\|x\|\|y\|}.
\]
Cauchy--Schwarz garantiza que $-1 \leq \cos(\theta) \leq 1$, 
haciendo válida la noción de similitud usada en NLP, visión y sistemas de recomendación.

\subsection*{6.2 Regularización y estabilidad}
En problemas de optimización se minimiza
\[
\min_w L(w) + \lambda \|w\|^2,
\]
donde la norma proviene de un producto interno. 
Cauchy--Schwarz asegura continuidad y estabilidad de los algoritmos de gradiente.

\subsection*{6.3 Kernels y SVMs}
Si $K(x,y)$ es un kernel en un espacio de Hilbert reproducing (RKHS),
\[
|K(x,y)| \leq \sqrt{K(x,x)} \, \sqrt{K(y,y)}.
\]
Esto valida el kernel como medida de similitud y sustenta métodos como SVMs y Gaussian Processes.

\subsection*{6.4 Acotación de errores}
Para variables aleatorias $X,Y$,
\[
|\mathbb{E}[XY]| \leq \sqrt{\mathbb{E}[X^2]} \, \sqrt{\mathbb{E}[Y^2]}.
\]
Este bound se usa en análisis de varianza, consistencia de estimadores 
y teoría de generalización (PAC learning).

\subsection*{6.5 Gradiente y optimización}
En descenso de gradiente,
\[
|\langle \nabla f(w), d\rangle| \leq \|\nabla f(w)\| \cdot \|d\|.
\]
Esto garantiza que el gradiente actúa como un funcional lineal continuo 
y que el paso en dirección $d$ está controlado.

\subsection*{6.6 PCA y reducción de dimensionalidad}
En PCA se maximiza
\[
\max_{\|v\|=1} v^T \Sigma v,
\]
donde $\Sigma$ es la matriz de covarianza. 
El máximo se alcanza en un autovector principal gracias a Cauchy--Schwarz, 
lo que fundamenta la reducción de dimensionalidad.

\bigskip
\noindent
\textbf{Conclusión:}  
La desigualdad de Cauchy--Schwarz puede verse como la afirmación de que la parábola
\[
Q(\lambda) = \|f - \lambda g\|^2
\]
\textbf{nunca desciende bajo el eje horizontal}.  
De esta manera, Cauchy--Schwarz no sólo es una desigualdad algebraica, 
sino también una expresión de la \emph{continuidad geométrica} de la proyección ortogonal.

\end{document}
